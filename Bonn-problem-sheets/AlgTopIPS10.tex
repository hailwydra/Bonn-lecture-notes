\documentclass[10pt,a4paper]{article}
\usepackage{preamble}
\title{Algebraic Topology I PS 10}
\author{Mathieu Wydra}
\date{}
\begin{document}
\maketitle
\begin{enumerate}
\item \begin{enumerate}
\item Let $\gamma^{1,n+1}_\R$ be the tautological lines bundle over $\R P^n$ .
\item 
\end{enumerate}
\item \begin{enumerate}
\item Let $\xi$ be a n-vector bundle $p:E\to B$ such that $B$ is compact. We first show $E$ is Hausdorff. Pick $x,y\in E$ such that $x\neq y$. If $p(x) = p(y) = b$, then the trivialising neighbourhood $\U\subset B$ of $b$ is such that $x,y\in p^{-1}(\U)\cong \U\times \F^n$. Since $\U\times \F^n$ is also Hausdorff the result follows. For local compactness, note that for $\U\subset B$ an open subset, $\U$ is locally compact as $B$ compact (so locally compact). Then any point $x\in E$ is contained in $p^{-1}(\U)\cong \U\times \F^n$ which is also locally compact, thus we have open $\V$ and compact $K$ of $p^{-1}(\U)$ such that $x\in \V\subset K$. Moreover, $K$ is compact in $E$ hence the result follows. Denote one-point compactification of $E$ as $E^+$.
\item Consider the map $D(E)\to E^+ = E\cup \{\infty\}$ defined fibrewise as $v \mapsto v/ \sqrt{1-\vert v\vert^2}$ on $D(E)\backslash S(E)$ and maps $S(E)\subset D(E)$ to $\infty$. This is evidently continuous as a map from $S(E)\subset D(E)$ to $E$. Also on fibres it sends $D(E)_b\xleftrightarrow{1:1} E_b$ and so is a bijection from $D(E)/S(E)$ to $E^+$.
\end{enumerate}
\end{enumerate}
\end{document}