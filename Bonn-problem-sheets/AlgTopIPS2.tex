\documentclass[10pt,a4paper]{article}
\usepackage{preamble}

\title{Algebraic Topology I Problem Sheet 2}
\author{Mathieu Wydra}
\begin{document}
\maketitle

\begin{enumerate}
\item
\begin{enumerate}
\item Suppose $S^n\to E\to B$ is a fibre sequence, $n\neq 0$ and $B$ simply connected. By Serre's theorem there exists a Serre spectral sequence 
\[E^2_{p,q} = H_p(B; H_q(S^n;\Z))\implies H_{p+q}(E;\Z).\]
Since $H_q(S^n) = \begin{cases}
\Z & p=0,n\\
0 & \text{else}
\end{cases},$, then $E^2_{p,q} = H_p(B; H_q(S^n;\Z)) = H_p(B;\Z)$ for $p=0,n$ and vanishes for other $p$. The $E^2$-page is given by

\begin{sseqdata}[name = one, homological Serre grading, classes = {draw = none}, no y ticks, y range = {0}{6}, y axis gap = 1cm, y tick gap = 1cm]

\begin{scope}[ background ]
\node at (\xmin - 1,0) {0};
\node at (\xmin - 1,1) {1};
\foreach \n in {2,6}{
\node at (\xmin - 1,\n) {\vdots};
}
\node at (\xmin - 1,3) {n-1};
\node at (\xmin - 1,4) {n};
\node at (\xmin - 1,5) {n+1};

\end{scope}

\foreach \x in {0,...,3} {\class["H_{\x}(B)"](\x ,0)}
\foreach \x in {0,...,3} \foreach \n in {1,3,5}{\class["0"](\x ,\n)}
\foreach \x in {0,...,3} \foreach \n in {2,6}{\class["\vdots"](\x ,\n)}
\foreach \x in {0,...,3} {\class["H_{\x}(B)"](\x ,4)}
\end{sseqdata}
\printpage[name = one,page = 2]\\
The only differentials are $d^{n+1}$ given by $d^{n+1}:H_p(B;\Z)\mapsto H_{p-n-1}(B,\Z)$ and $E^\infty = E^{n+2}$.

Since $E^{n+2}\cong H_*(E^{n+1})$, we get an exact sequence
\[0 \to E^\infty_{p,0}\to H_{p}(B) \xrightarrow{d^{n+1}} H_{p-n-1}(B)\to E^\infty_{p-n-1,n}\to 0\]

We have that $E^\infty_{p-k,k} = F^{p-k}_p/F^{p-k-1}_p = 0$ for $k=1,\ldots, n-1$ so $E^\infty_{p,0} = \left(F^p_p/F^0_p\right)\ldots \left(F^{p-n+1}_p/F^{p-n}_p\right) = F^p_p/F^{p-n}_p$. Moreover, $E^\infty_{p-n-k, n+k} = F^{p-n-k}_p/F^{p-n-k-1}_p = 0$ for all $k=1,2,\ldots$ so take it up to $k=p-n$, then combining as before $E^\infty_{p-n, n} = F^{p-n}_p$. Note $F^p_p= H_p(E;\Z)$. Thus there is an exact sequence
\[0\to E^\infty_{p-n,n} \to H_p(E;\Z)\to E^\infty_{p,0}\to 0\]

Combining this with the previous sequence give us a long exact sequence

\[\ldots\to H_{p-n}(B)\to H_p(E)\to H_p(B)\to H_{p-n-1}(B)\to H_{p-1}(E)\to \ldots\]
Exactness is obvious from the exactness of the two short exact sequences and identification of $E^\infty	_{p,q}$.

\item Similarly, suppose $F\xrightarrow{i} E\to S^n$ is a fibre sequence, $n\neq 0,1$.
By Serre's theorem there exists a Serre spectral sequence 
\[E^2_{p,q} = H_p(S^n; H_q(F;\Z))\implies H_{p+q}(E;\Z).\]

Since $H_p(S^n) = \begin{cases}
\Z & p=0,n\\
0 & \text{else}
\end{cases},$
and $E^2_{p,q} = H_p(S^n; H_q(F;\Z)) = H_q(F;\Z)$ by the universal coefficient theorem,

then the $E^2$-page is 
\begin{sseqdata}[name = two, homological Serre grading, classes = {draw = none}, no x ticks, x range = {0}{6}, y axis gap = 1cm ]

\begin{scope}[ background ]
\node at (0,\ymin - 1) {0};
\node at (1,\ymin - 1) {1};
\foreach \n in {2,6}{
\node at (\n,\ymin - 1) {\ldots};
}
\node at (3,\ymin - 1) {n-1};
\node at (4,\ymin - 1) {n};
\node at (5,\ymin - 1) {n+1};

\end{scope}

\foreach \y in {0,...,2} {\class["H_{\y}(F)"](0,\y)}
\foreach \y in {0,...,2} \foreach \n in {1,3,5}{\class["0"](\n,\y)}
\foreach \y in {0,...,2} \foreach \n in {2,6}{\class["\ldots"](\n,\y)}
\foreach \y in {0,...,2} {\class["H_{\y}(F)"](4,\y)}
\end{sseqdata}

\printpage[name = two,page = 2]\\
Hence the only differentials are $d^n$ given by $d^n:H_q(F)\mapsto H_{q+n-1}$ and $E^\infty = E^{n+1}$.

Since $E^{n+1}\cong H_*(E^n)$, we get an exact sequence
\[0 \to E^\infty_{n,q-n}\to H_{q-n}(F) \xrightarrow{d^n} H_{q-1}(F)\to E^\infty_{0,q-1}\to 0\]

We have that $E^\infty_{k,q-k}\cong F^k_q/F^{k-1}_q =0$ for all $k = 1,2,\ldots,n-1$, so $E^\infty_{n,q-n}\cong F^n_q/F^0_q$ and since $F^n_q = H_q(E;\Z)$, $E^\infty_{0,q} = F^0_q$ we also have an exact sequence \[0\to E^\infty_{0,q} \to H_q(E;\Z)\to E^\infty_{n,q-n}\to 0\]

Combining the two sequences we get a long exact sequence
\[\ldots\to H_q(F;\Z)\xrightarrow{i_*} H_q(E;\Z)\to H_{q-n}(F;\Z)\to H_{q-1}(F;\Z)\to H_{q-1}(E;\Z)\xrightarrow{i_*} \ldots\]



\end{enumerate}
\end{enumerate}
\end{document}