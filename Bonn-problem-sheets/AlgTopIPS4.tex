\documentclass[10pt,a4paper]{article}
\usepackage{preamble}
\title{Alg Geo I PS 4}
\author{Mathieu Wydra}
\date{}
\begin{document}
\maketitle
\begin{enumerate}
\item 
Firstly we compute the Eilenberg-MacLane space $K(\Z_2,3)$. To do that first we look at the path-loop fibration $K(\Z_2,1)\to *\to K(\Z_2,2)$. For simplicity we denote $K_n = K(\Z_2,n)$. As $K_1 = \R P^\infty$, the first few homology and cohomology groups are as follows, 
\begin{align*}
H_*(K_1;\Z) &= \Z,\Z_2,0,\Z_2,0,\Z_2,0,\Z_2\\
H^*(K_1;\Z) &= \Z,0,\Z_2,0,\Z_2,0,\Z_2,0\\
H_*(K_1;\Z_2) &= \Z_2,\Z_2,\Z_2,\Z_2,\Z_2,\Z_2\\
H^*(K_1;\Z_2) &= \Z_2,\Z_2,\Z_2,\Z_2,\Z_2,\Z_2\\
H_*(K_2;\Z) &= \Z,0,\Z_2,0\\
H^*(K_2;\Z) &= \Z,0,0,\Z_2\\
H_*(K_2;\Z_2) &= \Z_2,0\\
H^*(K_2;\Z_2) &= \Z_2,0
\end{align*}


We know that $H^*(K_1;\Z_2)\cong \Z_2[x]$ the polynomial ring on one generator. Thus by UCT we can see that $H^p(K_2;H^q(K_1;\Z_2)) = H^p(K_2;\Z_2)\otimes H^q(K_1;\Z_2)$.
The cohomological Serre spectral sequence on $\Z_2$ of the fibre sequence, 

\begin{sseqdata}[name = K_2_2, cohomological Serre grading, classes = {draw = none}, no x ticks, no y ticks, y range = {0}{5}]
\foreach \y in {0,...,5} {\class["\Z_2"](0,\y)}
\foreach \y in {0,...,5} {\class["\Z_2"](2,\y)}
\foreach \y in {0,...,5} {\class["H^3"](3,\y)}
\foreach \y in {0,...,5} {\class["H^4"](4,\y)}
\foreach \y in {0,...,5} {\class["H^5"](5,\y)}
\foreach \x in {1} \foreach \y in {0,...,\ymax} {\class["0"](\x,\y)}
\d2(0,1)
\d2(0,3)
\d2(0,5)
\d2(2,1)
\d2(2,3)
\end{sseqdata}
\printpage[name = K_2_2,page = 2]\\
Take the generator $x\in E_2^{0,1}$ of $H^*(K_1;\Z_2)$, $x$ is a transgressive element and by vanishing at $E_\infty$ page, $H^2(K_2;\Z_2)= \Z_2$ with generator $y = d_2(x)$. Then $x^2\in E_2^{0,2}$, $x^4\in E_2^{0,4}$ are both transgressive hence $d_2(x^2)=d_2(x^4) = 0$ and $d_2(x^3) = y x^2$, $d_2(x^5) = y x^4$. Also the multiplicative structure $E_2^{2,q} = \Z_2$ is generated by $y x^q$. Hence $E_2^{2,2}$, $E_2^{2,4}$ vanish at the $E_\infty$ page. By the transgression, $E_2^{2,1}$, $E_2^{2,3}$, $E_2^{2,5}$ do not vanish at the $E^3$ page, but $d_2(yx) = y^2$, $d_2(yx^3) = y^2 x^2$, $d_2(yx^5) = y^2 x^4$. In particular $E_2^{4,0}$ has a subgroup generated by $y^2$.
 Note that $y$ has degree 2, so we take the 2nd order Steenrod square such that $\text{Sq}^2(y) = y^2$. Similarly, $x$ has degree one and so $\text{Sq}^1(x) = x^2$ so $z = d_3(x^2) = \text{Sq}^1(d_3(x)) = \text{Sq}^1(y)$ and hence again $E_2^{3,0}$ contains $\Z_2$ generated by $\text{Sq}^1(y)$. By a similar process, $E_2^{5,0}$ is $\Z_2$ generated by $\text{Sq}^2\text{Sq}^1(y)$.
 \\

For the integral co-homological sequence;
\begin{sseqdata}[name = K_2, cohomological Serre grading, classes = {draw = none}, no x ticks, no y ticks, y range = {0}{6}]
\class["\Z"](0,0)
\class["\Z_2"](0,2)
\class["\Z_2"](0,4)
\class["\Z_2"](0,6)
\class["\Z_2"](3,0)
\class["\Z_2"](3,2)
\class["H^5"](5,0)
\class["H^6"](6,0)
\foreach \x in {1,2,4} \foreach \y in {0,...,\ymax} {\class["0"](\x,\y)}
\foreach \x in {0,3,5,6} \foreach \y in {1,3,5} {\class["0"] (\x,\y)}
\d["d_3"]3(0,2)
\end{sseqdata}
\printpage[name = K_2,page = 3]

Which converges to the $E^\infty$ page which vanishes everywhere except $E^\infty_{0,0} = \Z$. Note that $E^2_{4,0}$ vanishes as there is no-nontrival differential to it. Looking at the mulitplicative structure, the generator $x\in E^2_{0,2} =\Z_2$ maps to the generator $y = d_3(x)\in \Z$ and $d_3(x^2) = 2xy$ so $d_3:E^2_{0,4}\to E^2_{3,2}$ is not an isomophism and so for vanishing at the $E^\infty$ page $H^6$ must be non-empty. Note further that $d_3(xy) = d_3(x)y = y^2$ so it is clear that $H^6 =\Z_2$.






From UCT we have an exact sequence $0\to \Z_2\to H^3(K_2;H^2(K_1)) = H^3(K_2;\Z_2)\to \Z_2\to 0$. Also there is a $d_5$ differential $d_5:\Z_2\to H^5(K_2)$ so $H^5(K_2)=\Z_4$ . From this we can find the homology groups $H_*(K_2) = \Z,0,\Z_2,0,\Z_2,\Z_2$.\\
Similarly we do this for $K_2\to *\to K_3$. We already have that $H_*(K_3) = \Z,0,0,\Z_2,0$, $H^*(K_3) = \Z,0,0,0,\Z_2,0$. Again as before $H^4(K_3;H^3(K_2)) = \Z_2\otimes \Z_2$


\begin{sseqdata}[name = K_3, cohomological Serre grading, classes = {draw = none}, no x ticks, no y ticks]
\class["\Z"](0,0)
\class["\Z_2"](4,0)
\class["\Z_2"](0,3)
\class["\Z_2"](0,5)
\class["\Z_2"](0,6)
\class["\Z_2\otimes \Z_2"](4,3)
\foreach \x in {1,2,3,5} \foreach \y in {0,...,\ymax} {\class["0"](\x,\y)}
\foreach \x in {0,4} \foreach \y in {1,2,4} {\class["0"] (\x,\y)}

\end{sseqdata}
\printpage[name = K_3,page= 2]\\

Similarly to before we get that the only differential hitting $E_2^{0,5}$ is $d_6:E_2^{0,5} = \Z_2\to E_2^{6,0}$ and since both these vanish at the $E_\infty$ page, it is an isomrphism, thus $H^6(K_3,\Z_2) = \Z_2$. Finally, we can apply UCT again to get homology groups $H_*(K_3) = \Z,0,0,\Z_2,0,\Z_2$.\\
Since $S^3$ is CW and 2-connected, there exists a Whitehead tower up to degree 4
\[\begin{tikzcd}
K(\Z_2,3)\arrow{d}&K(\Z,2)\arrow{d}&\\
X_4\arrow{r}& X_3\arrow{r}& X_2=S^3
\end{tikzcd}
\]
By long exact sequences on the fibrations $K(\Z_2,3)\to X_4\to X_3$ and $K(\Z,2)\to X_3\to S^3$, $\pi_5(S^3) = \pi_5(X_4)$ and by Hurewicz $\pi_5(X_4) = H_5(X_4)$ so it suffices to compute the fifth homology group of $X_4$.
First we look at $X_3$, i.e the fibration $K(\Z,2)\to X_3\to S^3$. Note $\C P^\infty = K(\Z,2)$ so $H^q(K(\Z,2)) = \Z,0,\Z,0,\Z,0\ldots$. Then cohomological Serre spectral sequence has $E_2$-page $E_2^{p,q} = H^p(S^3, H^q(K(\Z,2)))\implies H^{p+q}(X_3)$

\begin{sseqdata}[name=X_3, cohomological Serre grading, classes = {draw = none}, no x ticks, no y ticks]
\foreach \x in {0,3} \foreach \y in {0,2,4,6} {\class["\Z"](\x,\y)}
\foreach \x in {1,2,4} \foreach \y in {0,...,\ymax} {\class["0"](\x,\y)}
\foreach \x in {0,3} \foreach \y in {1,3,5} {\class["0"](\x,\y)}
\d3(0,2)
\d3(0,4)
\d3(0,6)
\end{sseqdata}
\printpage[name=X_3, page=3]\\
Non-trivial differentials are all $d_3$ then if $x\in E_2^{0,2} = H^2(K(\Z,2)) = \Z$ is a generator then $y = d_3(x)$ is a generator for $E_2^{3,0} = H^3(S^3)$. By the multiplicative structure $xy$ generates $E_2^{3,2}$ and as $d_3(x^2)= d_3(x)x + xd_3(x) = 2xy$ so $x^2$ is twice a generator of $E_2^{0,4} = H^4(K(\Z,2))$. Similarly $x^3$ is thrice a generator of $H^6(K(\Z,2))$ since $d_3(x^3) = 3x^2y$.\\
As there are no more larger degree differentials after this $E_4 = E_\infty$, and so the first four cohomology group are easily seen to be $H^n(X_3) = \Z,0,0,0$ and due to the maps $\Z\xrightarrow{\times 2, \times 3}\Z$, $H^4(X_3) = 0$, $H^5(X_3) = \Z_2$, $H^6(X_3)=0$ and $H^7(X_3)=\Z_3$. Then by UCT we see that the first few homology groups are $H_n(X_3) = \Z,0,0,0,\Z_2,0,\Z_3$.\\
Now looking at the homological Serre spectral sequence for $K(\Z_2,3)\to X_4\to X_3$.

\begin{sseqdata}[name=X_4,homological Serre grading, classes = {draw = none}, no x ticks, no y ticks]
\class["\Z"](0,0)
\class["\Z_2"](4,0)
\class["\Z_3"](6,0)
\class["\Z_2"](0,3)
\class["\Z_2"](0,5)
\foreach \x in {1,2,3,5} \foreach \y in {0,...,\ymax} {\class["0"](\x,\y)}
\foreach \x in {0,4,6} \foreach \y in {1,2,4} {\class["0"](\x,\y)}
\d["d_6"]6(6,0)
\end{sseqdata}
\printpage[name= X_4,page = 6]

Since the only morphism $\Z_3 \to \Z_2$ is the trivial map, it follows that at the $E^\infty$-page $H_5(X_4) = E^\infty_{0,5} = \Z_2$. Hence $\pi_5(S^3) = \Z_2$.

\item Let $X = \text{map}(S^1,S^3)$, take base-point $*\in S^1$ and consider the evaluation map $\text{ev}_*:X\to S^3$, $f\mapsto f(*)$. Take base-point $b\in S^3$ and fibre sequence $F:=\text{ev}_*^{-1}(b)\to X\xrightarrow{\text{ev}_*}S^3$. Note that $F$ is just the loop space on $S^3$. Every map $S^1\to S^3$ is null-homotopic, $X$ is path connected. From the fibre sequence, there is a long exact sequence of homotopy groups. Since $\pi_0(S^3) = \pi_1(S^3) = \pi_2(S^3)= 0$ and the loop-space $F$ is simply connected (in fact contractible), $\pi_1(X) = 0$. Looking at the long exact sequence
\[\ldots\to\pi_3(F)\to \pi_3(X)\to \pi_3(S^3) = \Z\to\pi_2(F)\to\pi_2(X)\to 0\to\ldots\]
$\text{ev}_0$ has a right inverse $\sigma: S^3\to X$ taking $x\in S^3$ to the constant map to $x$. Hence the map $\pi_3(X)\to \pi_3(S^3)$ is surjective, so $\pi_3(S^3)\to \pi_2(F)$ is the zero map. Therefore $\pi_2(X) = \pi_2(F)$ (also follows from splitting) and as $F \cong \Omega(S^3)$, $\pi_2(\Omega(S^3)) \cong \pi_3(S^3) = \Z$. By Hurewicz then $H_2(X) = H_2(F) = \Z$ and $H^2(X)= H^2(F) = \Z$ from UCT. We know that $H^*(F)$ is the divided power algebra on one generator $\Gamma(x)$ so looking at the cohomological Serre spectral sequence, 

\begin{sseqdata}[name=X, cohomological Serre grading, classes = {draw = none}, no x ticks, no y ticks]
\foreach \x in {0,3} \foreach \y in {0,2,4} {\class["\Z"](\x,\y)}
\foreach \x in {1,2,4} \foreach \y in {0,...,5} {\class["0"](\x,\y)}
\foreach \x in {0,3} \foreach \y in {1,3,5} {\class["0"](\x,\y)}
\d3(0,2)
\d3(0,4)
\end{sseqdata}
\printpage[name =X,page=3]\\
We see the only non-trival differentials are $d_3:H^0(S^3;H^{2k}(F)) \to H^3(S^3;H^{2k-2}(F))$. The case $k=1$ must be the zero map as $H^3(X) \cong E_\infty^{3,0}\cong H^3(S^3)/\text{im}(d_3)$ so the map $H^3(S^3)\to H^3(X)$ induced by the right inverse from before is injective (since it has left inverse $\text{ev}_*$).\\
Then as $H^*(F)$ is the divided power algebra on one generator $\Gamma(x)$, the multiplicative structure shows inductively that all $d_3$ are zero maps.
Hence we get
\[H^n(X) = \begin{cases} 0 & \text{ for } n=1\\
 \Z & \text{ else }
 \end{cases}\]
\end{enumerate}
\end{document}